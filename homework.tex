1.何謂HTML
WWW上製作網頁(HomePage)的語言,是一種標籤語言(Tag Language),可以讓瀏覽器處理及呈現原始文件的內容。
2.何謂XHTML
3.何謂CSS
CSS (CASCADING STYLE SHEET) 中文翻譯為樣式表,是W3C組織所擬定出的一套標準格式。
W3C有鑑於網際網路的日益發達,網頁製作更是琳瑯滿目,之前的HTML漸不敷所需,也無法給網頁設計者較大的自由設計空間,於是乃訂定出CSS。現在網頁設計頗有名聲的DHTML(動態網頁設計),CSS便在裡面扮演重要的角色。
一般的情況下,如使用市面上的網頁編輯軟體來編輯設計的話,編輯基本的網頁已是足夠的了,但其在設計配置的自由程度上並不是那麼的方便順心,並無法隨心控制網頁元素的需求要素。
如使用CSS樣式表來作編排,由於CSS有絕對位置和相對位置的概念,可以計設出多層次和陰影等效果及其他特效。 
實際上CSS是一組網頁建構基礎樣式,CSS 主要增加了更多的樣式定義方式來輔助 HTML語言,透過 CSS樣式表的運作後,只要修改定義標籤(如:表格、背景、連結、文字、按鈕、Scroll bar ...等)的樣式,則其它各網頁相同標籤引用相同的樣式表檔案,所有引用檔案的網頁會呈現同一的引用樣式,建立一個風格統一的網站。
簡而言之,CSS 僅是在既有的網頁HTML語言中,加上一些輔助語法(Plugins),藉以達到控制網頁外觀的另一個簡便的方式。
4.何謂Javascript
JavaScript是網景公司所發展出來的一種跨平台語言,也是一種物件導向式的網頁式手稿語言。 
它是一種簡單的程式,由ASCII的字元來構成,可於記事本等文書編輯軟體來開發完成;一般程 式語言必須經過編譯之手續,但JavaScript並不需經過此道手續,它只要透過適當的直譯器 (browser)即可轉譯並執行。在個人電腦功能越來越強的今天,JavaScript也因此而大行其道, 畢竟在有限的頻寬下做最有效率的運作是使用者所樂見的。 
JavaScript雖是一簡單的語言,但是它卻有很強大的功能;特別是用於製作網頁特效、提供表單 前端驗證、視窗動態操作、事件驅動………等。 
5.何謂jQuery
jQuery 是一個快速又簡潔的 JavaScript 程式庫,簡化了在 HTML 文件裡面尋找 DOM 物件、處理事件、製作動畫、和處理 Ajax 互動的過程。
jQuery 對瀏覽器的支援
雖然各個瀏覽器因各自的特性對網頁所產生出來的效果不盡相同,但是 jQuery 大部分均能將其結果依照各個瀏覽器的性質來呈現,也就是說 jQuery 是跨瀏覽器的。
6.何謂jQuery easyui
7.何謂PHP
PHP是開放源碼的通用腳本語言,特別適合用來開發網站程式,可以內嵌在HTML碼。
PHP程式的原始碼是純文字,所以可以用任何可處理純文字檔的文字編輯器,
如:記事本、vi、emac等,來撰寫PHP程式。
8.何謂MySQL
9.何謂DOM
10.何謂Ajax
11.何謂JSON
12.哪些是語法,哪些是程式語言,其餘的又是什麼
13.動態網頁與靜態網頁的差別

8/5作業
簡單的上網收尋一下
各大行業的證照需求以及門檻
建議從人力銀行先收尋
104,YES123,518,1111,MyJOB等等...
大致搜尋一下即可
(討論重點:了解目前各大行業對證照需求)

